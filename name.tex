\documentclass[11pt,letterpaper]{article}

\usepackage{showlabels}
\usepackage{fullpage}
\usepackage{pslatex}
%\usepackage{latexsym}
\usepackage[english]{babel}
\usepackage[utf8]{inputenc}
\usepackage{amsmath}
\usepackage{bm}
%\usepackage{tikz}
\usepackage{xcolor}
\usepackage{url}
%\usepackage[colorinlistoftodos]{todonotes}
\usepackage{rotating}
\usepackage{natbib}
\usepackage{amssymb}
\usepackage{lingmacros}


%\usepackage{latexsym}
\usepackage[english]{babel}
\usepackage[utf8]{inputenc}
\usepackage{bm}
\usepackage{graphicx}
%\usepackage{tikz}
\usepackage{xcolor}
\usepackage{url}
%\usepackage[colorinlistoftodos]{todonotes}
\usepackage{rotating}
\usepackage{multirow}


%\usepackage{linguex}
%\usepackage{lingmacros}


\usepackage{hyperref}

\usepackage{tikz-dependency}
\usepackage{changepage}
\usepackage{longtable}


\newcommand{\R}[0]{\mathbb{R}}
\newcommand{\Prob}[0]{\mathbb{P}}
\newcommand{\Ff}[0]{\mathcal{F}}

\usepackage{multirow}

\newcommand{\soft}[1]{}
\newcommand{\nopreview}[1]{}
\newcommand\comment[1]{{\color{red}#1}}
\newcommand\mhahn[1]{{\color{red}(#1)}}
\newcommand\note[1]{{\color{red}(#1)}}
\newcommand\jd[1]{{\color{red}(#1)}}
\newcommand\rljf[1]{{\color{red}(#1)}}
\newcommand{\key}[1]{\textbf{#1}}

\DeclareMathOperator*{\argmax}{arg\,max}
\DeclareMathOperator*{\argmin}{arg\,min}
\DeclareMathOperator{\E}{\mathop{\mathbb{E}}}



\usepackage{amsthm}

\newcommand{\thetad}[0]{{\theta_d}}
\newcommand{\thetal}[0]{{\theta_{LM}}}

\newcounter{theorem}
\newtheorem{proposition}[theorem]{Proposition}
\newtheorem{thm}[theorem]{Theorem}
\newtheorem{corollary}[theorem]{Corollary}
\newtheorem{question}[theorem]{Question}
\newtheorem{example}[theorem]{Example}
\newtheorem{defin}[theorem]{Definition}
\newtheorem{definition}[theorem]{Definition}
\newtheorem{lemma}[theorem]{Lemma}


%\usepackage{linguex}
%\newcommand{\key}[1]{\textbf{#1}}



\renewcommand{\thefigure}{S\arabic{figure}}
\renewcommand{\thetable}{S\arabic{table}}
\renewcommand{\thesection}{S\arabic{section}}

\newcommand{\utterance}{\mathcal{U}}
\newcommand{\tree}{\mathcal{T}}



\usepackage{siunitx}



\usepackage{longtable}



\frenchspacing
%\def\baselinestretch{0.975}

%\emnlpfinalcopy
%\def\emnlppaperid{496}

\title{Morpheme Ordering across Languages reflects optimization for memory efficiency / information locality}

\begin{document}

\maketitle

\begin{abstract}
    The order of morphemes in a word shows well-documented tendencies across languages.
    These have been explained in terms of notions such as semantic scope and relevance in prior work.
    We propose an interpretation of these universals in terms of memory efficiency in incremental processing.
    The recently proposed Memory--Surprisal Tradeoff (CITE) provides an architecture-independent formalization of memory efficiency in incremental processing, and it has been proposed that optimization for efficiency of this tradeoff partly accounts for word order patterns in two languages.
    Efficient memory--surprisal tradeoffs are achieved by orders that display information locality, whereby elements that have high mutual information are closer together.
    In this work, we examine corpus data from six languages to show that optimizing for tradeoff efficiency mostly predicts morpheme order in verb and noun morphology.
    
\end{abstract}



\section{Introduction}

\section{Morpheme Ordering and Ordering Universals}

Verbs

Nouns


\section{Background: Memory--Surprisal Tradeoff and Information Locality}

CITE introduced the Memory--Surprisal Tradoff as an architecture-independent formalization of memory efficiency in incremental processing.


We formalize these three observations into postulates intended to provide a simplified picture of what is known about online language comprehension. Consider a listener comprehending a sequence of words $w_1, \dots, w_t, \dots, w_n$, at an arbitrary time $t$.
\begin{enumerate}
    \item Comprehension Postulate 1 (Incremental memory). At time $t$, the listener has an incremental \key{memory state} $m_t$ that contains her stored information about previous words. The memory state is characterized by a \key{memory encoding function} $M$ such that $m_t = M(w_{t-1}, m_{t-1})$.
    \item Comprehension Postulate 2 (Incremental prediction). The listener has a subjective probability distribution at time $t$ over the next word $w_t$ as a function of the memory state $m_t$. This probability distribution is denoted $P(w_t|m_t)$.
    \item Comprehension Postulate 3 (Linking hypothesis). Processing a word $w_t$ incurs difficulty proportional to the \key{surprisal} of $w_t$ given the memory state $m_t$:
    \begin{equation}
    \label{eq:lossy-surp}
    \text{Difficulty} \propto -\log P(w_t | m_t).
\end{equation}
\end{enumerate}

\begin{definition}
The \key{memory--surprisal tradeoff curve} for a process $W$ is the lowest achievable average surprisal $S_M$ for each value of $H_M$. Let $R$ denote an upper bound on the memory entropy $H_M$; then the memory--surprisal tradeoff curve as a function of $R$ is given by
\begin{equation}
    \label{eq:ms-formal}
    D(R) \equiv \min_{M : H_M \le R} S_M,
\end{equation}
where the minimization is over all memory encoding functions $M$ whose entropy $H_M$ is less than or equal to $R$.
\end{definition}

Our argument will make use of a quantity called \key{mutual information}. Mutual information is the most general measure of statistical association between two random variables. The mutual information between two random variables $X$ and $Y$, conditional on a third random variable $Z$, is defined as:
\begin{align}
\label{eq:mi}
    \operatorname{I}[X:Y|Z] &\equiv \sum_{x,y,z} P(x,y,z) \log \frac{P(x,y|z)}{P(x|z)P(y|z)}. % \text{ bits} \\
    %\nonumber
    %&= \operatorname{H}[X|Z] - \operatorname{H}[X|Y,Z] \\
    %\nonumber
    %&= \operatorname{H}[Y|Z] - \operatorname{H}[Y|X,Z].
\end{align}
Mutual information is always non-negative. It is zero when $X$ and $Y$ are conditionally independent given $Z$, and positive whenever $X$ gives any information that makes the value of $Y$ more predictable, or vice versa. 

We will study the mutual information structure of natural language sentences, and in particular the mutual information between words at certain distances in linear order. We define the notation $I_t$ to mean the mutual information between words at distance $t$ from each other, conditional on the intervening words:
\begin{equation*}
    I_t \equiv \operatorname{I}[w_t : w_0 | w_1, \dots, w_{t-1}].
\end{equation*}
This quantity, visualized in Figure~\ref{fig:theorem}(a), measures how much predictive information is provided about the current word by the word $t$ steps in the past.
It is a statistical property of the language, and can be estimated from large-scale text data.

Equipped with this notion of mutual information at a distance, we can now state our theorem:
\begin{thm}\label{prop:suboptimal}(Information locality bound)
For any positive integer $T$, let $M$ be a memory encoding function such that
\begin{equation}
\label{eq:memory-bound}
H_M \le \sum_{t=1}^T t I_t.    
\end{equation}
Then we have a lower bound on the average surprisal under the memory encoding function $M$:
\begin{equation}
\label{eq:surprisal-bound}
S_M \ge S_\infty + \sum_{t=T+1}^\infty I_t.
\end{equation}
\end{thm}
\section{Methods}

\subsection{Data} % TODO: Becky

We considered:

- UD data for Japanese, Korean, Turkish, Hungarian, Finnish

- CHILDES data for Sesotho

The Sesotho and Japanese data was previously used by CITE; we reanalyze these data here in a way consistent across all six languages.


\subsection{Nouns}

\paragraph{Turkish Nouns}

\paragraph{Hungarian Nouns}

\paragraph{Finnish Nouns}


\subsection{Verbs}

\paragraph{Hungarian Verbs}

\paragraph{Turkish Verbs}

\paragraph{Finnish Verbs}

\paragraph{Sesotho Verbs}

\begin{enumerate}
    \item Subject agreement: This morpheme encodes agreement with the subject, for person, number, and noun class (the latter only in the 3rd person) \cite[\textsection 395]{doke1967textbook}.
            The annotation provided by \cite{demuth1992acquisition} distinguishes between ordinary subject agreement prefixes and agreement prefixes used in relative clauses; we distinguish these morpheme types here.

    \item Negation \citep[\textsection 429]{doke1967textbook}

    \item Tense/aspect marker   \citep[\textsection 400--424]{doke1967textbook}

    \item Object agreement or reflexive marker \citep[\textsection 459]{doke1967textbook}.
    Similar to subject agreement, object agreement denotes person, number, and noun class features of the object.
\end{enumerate}
We identified the following suffixes:

% something we might cite at some point (pointers from Beth Levine), about templatic morpheme order in Bantu
% Hyman2003 in literature.bib
% Jeffrey Good, Strong Linearity: Three Case Studies Towards a Theory of Morphosyntactic Templatic Constructions (Diss, 2003)

\begin{enumerate}
\item Semantic derivation: reversive (e.g., `do' $\rightarrow$ `undo') \citep[\textsection 345]{doke1967textbook}
\item Valence: Common valence-altering suffixes include causative, neuter/stative, applicative, and reciprocal \citep[\textsection 307--338]{doke1967textbook}. See SI for details on their meanings.
    \item Voice: passive \citep[\textsection 300]{doke1967textbook}
    \item Tense \citep[\textsection 369]{doke1967textbook}
    \item Mood \citep[\textsection 386--445]{doke1967textbook}
    \item Interrogative and relative markers, appended to verbs in certain interrogative and relative clauses \citep[\textsection 160, 271, 320, 714, 793]{doke1967textbook}.
    %-\textit{ng}.
    %The interrogative marker -\textit{ng} is an enclitized form of the interrogative `what'.
    %The relative marker -\textit{ng} is suffixed to verbs in relative clauses.
\end{enumerate}



\paragraph{Japanese Verbs}


\begin{enumerate}
\item \textit{suru}: obligatory suffix after Sino-Japanese words when they are used as verbs
\item Valence: causative (-\textit{ase}-) (\citet[142]{hasegawa2014japanese}, \citet[Chapter 13]{kaiser2013japanese})
\item Voice and Mood: passive (-\textit{are}-, -\textit{rare}-) (\citet[152]{hasegawa2014japanese}, \citet[Chapter 12]{kaiser2013japanese}) and potential (-\textit{e}-, -\textit{are}-, -\textit{rare}-) \citep[398]{kaiser2013japanese}
\item Politeness (-\textit{mas}-) \citep[190]{kaiser2013japanese}.
\item Mood: desiderative (-\textit{ta}-) \citep[238]{kaiser2013japanese}
\item Negation (-\textit{n}-)
\item Tense, Aspect, Mood, and Finiteness: past (-\textit{ta}), future/hortative (-\textit{yoo}) \citep[229]{kaiser2013japanese}, nonfiniteness (-\textit{te}) \citep[186]{kaiser2013japanese}
%\item Nonfiniteness: the suffix -\textit{te} derives a nonfinite form .
\end{enumerate}



\paragraph{Korean Verbs}

We then considered morphemes occurring at least 50 times:

\begin{enumerate}
    \item Root (Valency is not separated in the dataset)
    \item Derivation:
    
    ha (from hada), i (predicative)
    
    \item Honorific -s-
    \item Tense/Aspect: -ess- for past, -essess- for remote past, -keyss- for future
    \item Formality -p-
    \item Mood 1:
    
    -n-
    
    -ni-
    
    -ri-
    
    -deon-
    
    \item `Pragmatic Mood'
    
    -da-
    
    -ra-
    
    Interrogative -ka, -lkka, -nya
    
    -ji
    
    -eo (informal)
    
    -sida
    
    ....
    
    \item Polite -yo
    \item Conjunctive endings
    
    -go
    
    -seo
    
    and others
    
\end{enumerate}

%\ex.\ag. oa di rek a \\
%\textsc{subject.agreement} \textsc{object.agreement} buy \textsc{indicative} \\
%`(he) is buying (it)'  \citep{demuth1992acquisition} \label{ex:oadireka}
%\bg. o pheh el a \\
%\textsc{subject.agreement} cook \textsc{applicative} \textsc{indicative} \\
%`(he) cooks (food) for (him)'  \citep{demuth1992acquisition}
%\label{ex:ophehela}

\begin{tabular}{llllllllll}
1    & 3 & 4     & 5   & 6  & 7 & 8 \\
bara & sy & eots & eum & ni & da  & &  `wished'\\
wish & Honorific & Past & Formal & Indicative & Indicative \\
bara& & gess &  & &  eo & yo  &  `will wish' \\
wish  & &  Assertive && & informal & polite \\
\end{tabular}

%https://en.wiktionary.org/wiki/%EB%B0%94%EB%9D%BC%EB%8B%A4

\section{Results}

\section{Relation to previous accounts}

\subsection{Semantic Scope; Syntactic Structure; Historical Development}

We do not see our account as competing with these. Rather, it specifies tendencies for ordering that could be satisfied at different levels of description.

CARP template in Bantu

\subsection{Relevance; Proximity; Iconicity}

The relation  between mutual information and relevance

\subsection{Beyond Agglutionation}

\section{Discussion}




\section{Conclusion}


\end{document}






