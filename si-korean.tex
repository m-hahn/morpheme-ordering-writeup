Korean verb morphology is very complex, and there is no generally agreed-upon description in terms of morphemes and slots.
We extracted suffixes based on the annotation found in the corpus we used (Kaist corpus, \citet{chun2018building}) and the linguistic literature on Korean \citep{yeon2010korean}.
The Kaist corpus already provides segmentation into morphemes; we postprocessed it in two ways:
First, we made it more fine-grained by splitting morphemes that are merged in the annotation, and, second, we abstracted away consistently from allomorphy.
For instance, the Kaist corpus separately labels the segments \korean{ㅂ니다} \textit{-mnida} (as in \korean{합니다} \textit{hamnida} `does') and \korean{습니다} \textit{-seumnida} (as in \korean{했습니다} \textit{hatseumnida} `did').
These two segments actually are allomorphs, conditioned by the preceding material.
Furthermore, they can both be segmented into the formal marker \korean{ㅂ/습} (\textit{-p/-seup}, we label this underlying morpheme ``\textsc{p}$_5$'', see below for this notation), and the mood markers \korean{니} (\textit{-ni}, our \textsc{ni}$_6$) and \korean{다} (\textit{-da}, our \textsc{da}$_7$).
We therefore transform both segments into the abstract morpheme sequence \textsc{p}$_5$-\textsc{ni}$_6$-\textsc{da}$_7$ (see below for this notation).
We then partitioned the resulting morphemes into slots that make it possible to consistently describe the ordering of all forms encountered in the corpus.
We indexed morphemes by a small-caps representation of a stylized phonological representation (such as \textsc{p} for -\korean{ㅂ/습}  -\textit{p/seup}),\footnote{These transcriptions are purely conventional, we do not intend these to correspond to a theory of how underlying phonological forms are realized as surface phone strings.} with a subscript indicating their slot (such as \textsc{p}$_5$-\textsc{ni}$_6$-\textsc{da}$_7$).
With this procedure, we identified the nine slots described below:
\begin{enumerate}
    %. The root may include a valency suffix, which is not separated frem the root in the dataset. We did not attempt to separate v
    \item Derivation: The two derivational suffixes are \textit{ha} (\citep[4.1.2]{yeon2010korean}) and the predicative \textit{i}, whose function is similar to that of a copula \citep[4.1.4]{yeon2010korean}.
    
    \item Honorific \textsc{si}$_3$ \citep[4.3.2, 4.4.1]{yeon2010korean}
    \item Tense/Aspect suffixes include -\textsc{ess}$_4$ for past \citep[4.5.1.1]{yeon2010korean}, -\textsc{essess}$_4$ for remote past \citep[4.5.1.2]{yeon2010korean}, -\textsc{get}$_4$ for future \citep[4.5.2.1]{yeon2010korean}
    \item Formality \textsc{p}$_5$ (allomorphs include -\textit{p}-, -\textit{m}-, -\textit{seum}-,  \citep[4.3.2]{yeon2010korean})
    \item Mood 1: We partition Mood suffixes into two slots, as these can be combined into strings of combined mood suffixes. Frequent elements of the Mood I slot are -\textsc{n}$_6$-, -\textsc{ni}$_6$- \citep[4.3.2]{yeon2010korean}.
    
    \item Mood II: Frequent elements of the Mood II slot are declarative -\textsc{da}$_7$- \citep[4.3.2]{yeon2010korean}, command -\textsc{ra}-, interrogative -\textsc{ka}-, the suffix  -\textsc{ji} \citep[4.2.2-3]{yeon2010korean}, and informal -\textsc{eo}$_7$-.
    
    \item Polite -yo
    \item Conjunctive endings, such as -go, -seo,  and others
    
\end{enumerate}

We show morphemes occurring at least 50 times in slots 2-8 in Figure~\ref{tab:korean-frequent-morphemes}.
We do not provide the list of the (very numerous) conjunctive and nominalizing suffixes as these are not the focus of the typological generalization that we are interested in.

Additional morphemes that occur less than 50 times are placed into an UNKNOWN slot. %, this affects \textcolor{red}{TODO} \% of morpheme occurrences in the dataset.


%TODO 
%- Yeon 4.4.2.2 kkeo object honorific % 꺼   % does not seem to appear in Kaist corpus?


%Frequent morphemes:
\begin{table}
\resizebox{1\textwidth}{!}{
\begin{tabular}{llllllllll}
Slot & Morphs & Short & Description & Citation \\ \hline\hline
Derivation & \korean{하} & \textsc{ha}$_2$ &	  & \citep[4.1.2]{yeon2010korean}\\
& \korean{이} & \textsc{i}$_2$ &	 & \citep[4.1.4]{yeon2010korean} \\ \hline
Honorific & \korean{시} 	& \textsc{si}$_3$ & &\citep[4.3.2, 4.4.1]{yeon2010korean}\\\hline
	Tense/Asp. & \korean{었었} 	& \textsc{essess}$_4$ &  & \citep[4.5.1.2]{yeon2010korean} \\
&\korean{겠} 	& \textsc{get}$_4$ & assertive & \citep[4.5.2.1]{yeon2010korean}\\
& \korean{ㅆ} &\textsc{ess}$_4$ & past& \citep[4.5.1.1]{yeon2010korean}\\\hline
Formality & \korean{ㅂ} &\textsc{p}$_5$ &	 formal-polite &\citep[4.3.2]{yeon2010korean}\\\hline
	Mood 1& \korean{리} & \textsc{ri}$_6$ &	 \\ % I-guess-
	&\korean{니} 	&\textsc{ni}$_6$ & See Table~\ref{tab:korean-hada-1} & \citep[4.3.2]{yeon2010korean}\\
	&\korean{ㄴ} 	&\textsc{n}$_6$ &See Table~\ref{tab:korean-hada-1}  \\\hline
	Mood 2& \korean{시다} & \textsc{sida}$_7$ & 	Hortative, formal, polite \\
       & \korean{어} 	& \textsc{eo}$_7$ & Indicative, informal\\
	& \korean{자} & \textsc{ja}$_7$&	 Hortative & \citep[4.3.6.3]{yeon2010korean}\\
	& \korean{소} &\textsc{so}$_7$  &	See Table~\ref{tab:korean-styles} \\
	& \korean{ㄹ까} & \textsc{lkka}$_7$& Interrogative & \citep[8.9]{yeon2010korean} \\
	& \korean{오} & \textsc{o}$_7$  & See Table~\ref{tab:korean-styles} \\
	& \korean{지} & \textsc{ji}$_7$ &	    & \citep[4.2.2-3]{yeon2010korean}\\
	& \korean{까} 	& \textsc{ka}$_7$ & Interrogative & \citep[4.3.4, p. 175; p. 183]{yeon2010korean} \\
	& \korean{라} &\textsc{ra}$_7$  &	 command & \citep[4.3.6.4]{yeon2010korean} \\
	& \korean{다} & \textsc{da}$_7$ &	 Declarative & \citep[4.3.2]{yeon2010korean} &  \\\hline
	Polite & \korean{요} & \textsc{yo}$_8$&	 See Table~\ref{tab:korean-styles} \\
\end{tabular}
}
\caption{Frequent Korean verb suffixes in slots 1-8. The first column describes the nine slots. Second column: Representation of one allomorph in Hangeul. Third column: Identifier.}\label{tab:korean-frequent-morphemes}
\end{table}


Next, we explain how our segmentation corresponds to commonly described paradigms.
First, in Table~\ref{tab:korean-styles}, we describe the correspondence to the speech style system described by \citep[4.3.2]{yeon2010korean}.
Second, in Tables~\ref{tab:korean-hada-1}--\ref{tab:korean-hada-3}, we show the paradigm for the verb \korean{하다} \textit{hada} `to do' as described in Wiktionary.\footnote{ \texttt{https://en.wiktionary.org/wiki/\korean{하다}} (retrieved Septemgber 16, 2020).}
In Table~\ref{tab:kroean-itda}, we show the part of the paradigm of the verb \textit{ida} `to be' corresponding to Table~\ref{tab:korean-hada-1} (the other parts of the paradigm are analogous to those of \textit{hada}).
Note that none of these paradigms constitute exhaustive lists of all forms of these verbs; instead, they represent commonly used forms in a systematic paradigmatic representation.


\begin{table}
	\begin{center}
\begin{tabular}{l||l|l|l|llll}
            & Statement & Question  & Command    & Proposal    \\ \hline\hline
Formal      &  \korean{ㅂ니다} & \korean{ㅂ니까} & \korean{지-ㅂ-지오} & \korean{지-ㅂ-지다} \\ 
      &  -mnida & -mnikka  & -sipsio & -sipsida  \\ 
      &  -\textsc{p}$_5$-\textsc{ni}$_6$-\textsc{da}$_7$ & -\textsc{p}$_5$-\textsc{ni}$_6$-\textsc{kka}$_7$  & -\textsc{si}$_3$-\textsc{p}$_5$-\textsc{sio}$_7$ & -\textsc{si}$_3$-\textsc{p}$_5$-\textsc{sida}$_7$ \\ \hline
Polite      &  \multicolumn{4}{c}{\korean{아/어/요}}  \\
      &  \multicolumn{4}{c}{-a/eoyo}  \\
            & \multicolumn{4}{c}{-\textsc{eo}$_7$-\textsc{yo}$_8$} \\ \hline
Semi-Formal & \korean{오/소}   &           &   \korean{오}       &  \korean{-ㅂ-지다} \\
&  -o/so   &           & -o        & -p-sida \\
	&  -\textsc{o}$_7$/-\textsc{so}$_7$   &           & -\textsc{o}$_7$    & -\textsc{p}$_5$-\textsc{sida}$_7$ \\\hline
Familiar    &    \korean{네}    &  \korean{나/는가} &  \korean{게}      &    \korean{세} \\
            & -ne                  & -na/neunka & -ge & -se \\ 
	    & -\textsc{ne}$_7$                  & -\textsc{na}$_7$/-\textsc{neunka}$_7$ & -\textsc{ge}$_9$ & \textsc{-se}$_9$ \\ \hline
Intimate      &  \multicolumn{4}{c}{\korean{아/어}}  \\
      &  \multicolumn{4}{c}{-a/eo}  \\
            & \multicolumn{4}{c}{-\textsc{eo}$_7$} \\ \hline
Plain       &   \korean{다}     &  \korean{(느)냐} &  \korean{라}     &  \korean{자}\\
            &  -da    &  -(neu)nya & -ra      & -ja \\
	    & -\textsc{da}$_7$    &   -\textsc{nya}$_7$          & -\textsc{ra}$_7$      & -\textsc{ja}$_7$\\
\end{tabular}
	\end{center}
\caption{Correspondence between our morpheme segmentation and the speech style system described by \citep[4.3.2]{yeon2010korean}. In each cell, we provide the Hangul ending given by \citep{yeon2010korean}, a transliteration, and a representation in terms of underlying morphemes.}\label{tab:korean-styles}
\end{table}



\begin{table}
\begin{tabular}{llllllllll}
           &          &Formal non-polite & Informal non-polite & Informal polite & Formal polite \\ \hline \hline
\multirow{6}{*}{Indicative} & \multirow{3}{*}{Non-past} & \korean{한다} & \korean{해}  & \korean{해요}  & \korean{합니다}  \\
           &          & handa & hae & haeyo &  hamnida \\
           &          & -\textsc{n}$_6$-\textsc{da}$_7$ & -\textsc{eo}$_7$ & -\textsc{eo}$_7$-\textsc{yo}$_8$ &  -\textsc{p}$_5$-\textsc{ni}$_6$-\textsc{da}$_7$ \\
           &          & \korean{하+ㄴ다}  & \korean{하+어}    & \korean{하+어+요} & \korean{하+ㅂ니다} \\
            &          &  px+ef        &   pvg+ecs        & pvg+ef+jxf & pvg+ef\\
           \hline
           & \multirow{3}{*}{Past}     & \korean{했다}  & \korean{했어} & \korean{했어요}   & \korean{했습니다}  \\
           &      & haet-da &  haesseo &  haesseoyo  & haetseumnida \\
           &      & -\textsc{ess}$_4$-\textsc{da}$_7$ &  -\textsc{ess}$_4$-\textsc{eo}$_7$ &  -\textsc{ess}$_4$-\textsc{eo}$_7$-\textsc{yo}$_8$  & -\textsc{ess}$_4$-\textsc{p}$_5$-\textsc{ni}$_6$-\textsc{da}$_7$ \\
           &      &  \korean{하+었+다}  &  \korean{하+었+어} & \korean{하+었+어+요} & \korean{하+었+습니다}  \\
           &      &  ncpa+xsv+ep+ef   & px+ep+ef   &   px+ep+ef+jxf &     pvg+ep+ef\\
           \hline
\multirow{6}{*}{Interrogative} & \multirow{3}{*}{Non-past} & \korean{하느냐} & \korean{해}  & \korean{해요}  & \korean{합니까} \\
 &  & haneunya &  hae &  haeyo & hamnikka\\
 &  & -\textsc{nya}$_7$ &  -\textsc{eo}$_7$ &  -\textsc{eo}$_7$-\textsc{yo}$_8$ & -\textsc{p}$_5$-\textsc{ni}$_6$+KKA$_7$ \\
               && \korean{하+느냐} & & & \korean{하+ㅂ니까}    \\
              && px+ef & & & px+ef \\
 \hline
              & \multirow{3}{*}{Past} & \korean{했느냐} & \korean{했어} & \korean{했어요} & \korean{했습니까} \\
              &  & ha-n-neunya &  hae-sseo &  hae-sseo-yo &  haetseumnikka\\
              &  & -\textsc{ess}$_4$-\textsc{nya}$_7$ &  -\textsc{ess}$_4$-\textsc{eo}$_7$ &  -\textsc{ess}$_4$-\textsc{eo}$_7$+YO$_8$ &  -\textsc{ess}$_4$-\textsc{p}$_5$-\textsc{ni}$_6$-\textsc{kka}$_7$\\
              &  &  \korean{하+었+느냐}  & \korean{하+었+어} & \korean{하+었+어+요} & \korean{하+었+습니까} \\
              &  & +ep+ef  & +ep+ef            & px+ep+ef+jxf & pvg+ep+ef \\
              \hline
Hortative   && \korean{하자}  & \korean{해}  & \korean{해요}  & \korean{합시다}  \\
   && haja &  hae &  haeyo & hapsida \\
   && -JA$_7$ &  -\textsc{eo}$_7$ &  -\textsc{eo}$_7$-\textsc{yo}$_8$ & -\textsc{p}$_5$-\textsc{sida}$_7$ \\
   && \korean{+자} && \korean{하+어+요} & \korean{하+ㅂ시다}\\
   && +ef  && pvg+ef+jxf & ef \\
   \hline
Imperative  && \korean{해라, 하여라}  & \korean{해}  & \korean{해요}  & \korean{합시오}  \\
  && haera, hayeora &  hae &  haeyo &  hapsio \\
  && -\textsc{ra}$_7$ & -\textsc{eo}$_7$ & -\textsc{eo}$_7$-\textsc{yo}$_8$ & -\textsc{p}$_5$-\textsc{si}$_3$O \\
  && \korean{하+어라}, \korean{하+어라}  &  &    & \korean{+ㅂ시오} \\
  && pvg+ef,  pvg+ef                     &  &    &  +ef\\
  \hline
Assertive   && \korean{하겠다}  & \korean{하겠어}  & \korean{하겠어요}  & \korean{하겠습니다}  \\
   &&  hagetda & hagesseo &  hagesseoyo & hagetseumnida \\
   &&  -\textsc{get}$_4$-\textsc{da}$_7$ & -\textsc{get}$_4$-\textsc{eo}$_7$ &  -\textsc{get}$_4$-\textsc{eo}$_7$-\textsc{yo}$_8$ & \textsc{get}$_4$-\textsc{p}$_5$-\textsc{ni}$_6$-\textsc{da}$_7$ \\
&& \korean{하+겠+다}  & \korean{하+겠+어}   & \korean{하+겠+어+요} & \korean{하+겠+습니다}\\
&&  px+ep+ef          & pvg+ep+ef   & pvg+ep+ef+jxf & pvg+ep+ef\\
\hline
\end{tabular}
	\caption{Correspondence between our morpheme segmentation and the paradigm of \textit{hada} `to do' as described by Wiktionary and in the Kaist treebank. In each cell, we provide the Hangul  and transliterated (e.g., handa) forms given by Wiktionary, a representation in terms of underlying morphemes (e.g., -\textsc{n}$_6$-\textsc{da}$_7$), and -- where available -- how the form is segmented in the Kaist corpus (e.g., px+ef).}\label{tab:korean-hada-1} % (e.g., \korean{한다})\korean{하+ㄴ다} and 
\end{table}

\begin{table}
	\begin{center}
\begin{tabular}{llllllllll}
           &          &Formal non-polite & Informal non-polite & Informal polite & Formal polite \\ \hline \hline
Cause/Reason && \korean{해} & \korean{해서, 하여서} & \korean{하니} & \korean{하니까} \\
&& hae & haeseo, hayeoseo & hani & hanikka \\ 
&& -\textsc{eo}$_7$ & -\textsc{eoseo}$_9$ & -\textsc{ni}$_9$ & -\textsc{nikka}$_9$ \\
&& &\korean{하+어서, 하+어서} & \korean{하+니} & \korean{하+니까} \\
&& & pvg+ecs, xsm+ecs         & pvg+ecs & xsm+ef \\
\hline
Contrast && \korean{하지만} & \korean{하는데} & \korean{하더니} \\
 && hajiman & haneunde & haedoni & \\ 
 && -\textsc{jiman}$_9$ & -\textsc{neunde}$_9$ & -\textsc{deoni}$_9$ & \\ \hline
Conjunction && \korean{하고} \\
 && hago \\ 
	&& -\textsc{go}$_9$ \\ \hline
Condition && \korean{하면} & \korean{해야, 하여야} \\
&& hamyeon & haeya, hayeoya \\
	&& -\textsc{meyon}$_9$ & -\textsc{eoya}$_9$ \\ \hline
Motive && \korean{하려고} \\
 && haryeogo \\
	&& -\textsc{ryeogo}$_9$
\end{tabular}
	\end{center}
	\caption{Continuation of Table~\ref{tab:korean-hada-1}: Forms of \textit{hada} with a conjunctive ending.}\label{tab:korean-hada-2}
\end{table}



\begin{table}[]
    \centering
    \begin{tabular}{llllllllllllllllllllllllll}
           &          &Formal non-polite & Informal non-polite & Informal polite & Formal polite \\ \hline \hline
\multirow{2}{*}{Indicative} & Non-past & \korean{하신다} & \korean{하셔} & \korean{하세요, 하셔요} & \korean{하십니다} \\
                           && -\textsc{si}$_3$-\textsc{n}$_6$-\textsc{da}$_7$ & \textsc{si}$_3$-\textsc{eo}$_7$ & \textsc{si}$_3$-\textsc{eo}$_7$-\textsc{yo} & \textsc{si}$_3$-\textsc{p}$_5$-\textsc{ni}$_6$-\textsc{da}$_7$ \\
&& \korean{하+시+ㄴ다}       &   --   &    \korean{하+시+어+요}             &  \korean{하+시+ㅂ니다}        \\
\hline
&& \korean{하셨다} & \korean{하셨어} & \korean{하셨어요} & \korean{하셨습니다} \\
&& \textsc{si}$_3$-\textsc{ess}$_4$-\textsc{da}$_7$ & \textsc{si}$_3$-\textsc{ess}$_4$-\textsc{eo}$_7$ & \textsc{si}$_3$-\textsc{ess}$_4$-\textsc{eo}$_7$-\textsc{yo} & \textsc{si}$_3$-\textsc{ess}$_4$-\textsc{p}$_5$-\textsc{ni}$_6$-\textsc{da}$_7$ \\
& & \korean{하+셨+다}       &  --    &        --        &  \korean{하+셨+습니다}       \\
\hline
\multirow{2}{*}{Interrogative} & Non-past & \korean{하시느냐} & \korean{하셔} & \korean{하세요, 하셔요} & \korean{하십니까} \\
&& \textsc{si}$_3$-\textsc{nya}$_7$ & \textsc{si}$_3$-\textsc{eo}$_7$ & \textsc{si}$_3$-\textsc{eo}$_7$-\textsc{yo} & \textsc{si}$_3$-\textsc{p}$_5$-\textsc{ni}$_6$-\textsc{kka}$_7$ \\
& &   --     &   --   &   \korean{하+시+어+요}             &  --        \\
\hline
& Past & \korean{하셨느냐} & \korean{하셨어} & \korean{하셨어요} & \korean{하셨습니까} \\
&      & \textsc{si}$_3$-\textsc{ess}$_4$-\textsc{nya}$_7$ & \textsc{si}$_3$-\textsc{ess}$_4$-\textsc{eo}$_7$          & \textsc{si}$_3$-\textsc{ess}$_4$-\textsc{eo}$_7$-\textsc{yo} & \textsc{si}$_3$-\textsc{ess}$_4$-\textsc{p}$_5$-\textsc{ni}$_6$-\textsc{kka}$_7$ \\
& &        &      &                &          \\
\hline
\multirow{2}{*}{Imperative} && \korean{하시라} & \korean{하셔} & \korean{하세요} & \korean{하십시오} \\
&& -\textsc{si}$_3$-RA & -\textsc{si}$_3$-\textsc{eo}$_7$ & -\textsc{si}$_3$-\textsc{eo}$_7$-\textsc{yo} & -\textsc{si}$_3$-\textsc{p}$_5$-\textsc{si}$_3$O \\
&        &      & \korean{하+시+어+}               &   \korean{하+십시오}       \\
\hline
\multirow{2}{*}{Assertive} & & \korean{하시겠다} & \korean{하시겠어} & \korean{하시겠어요} & \korean{하시겠습니다} \\
&        &  -\textsc{si}$_3$-\textsc{get}$_4$-\textsc{da}$_7$     &  -\textsc{si}$_3$-\textsc{get}$_4$-\textsc{eo}$_7$              &  -\textsc{si}$_3$-\textsc{get}$_4$-\textsc{eo}$_7$-\textsc{yo} & -\textsc{si}$_3$-\textsc{get}$_4$-\textsc{p}$_5$-\textsc{ni}$_6$-\textsc{da}$_7$        \\
&& -- & -- & -- & -- \\
    \end{tabular}
    \caption{Continuation of Tables~\ref{tab:korean-hada-1}--\ref{tab:korean-hada-2}: Conjugation of hada with an honorific (\textsc{si}$_3$). We provide Hangul forms, morpheme sequences, and -- where available -- segmentations as given in the Kaist corpus.}
    \label{tab:korean-hada-3}
\end{table}

%itda \url{https://en.wiktionary.org/wiki/%EC%9E%88%EB%8B%A4#Korean}

\begin{table}[]
    \centering
    \begin{tabular}{llllllllllllllllllllllllllllllllll}
           &          &Formal non-polite & Informal non-polite & Informal polite & Formal polite \\ \hline \hline
    \multirow{6}{*}{Indicative}         & Non-past  & \korean{있다} & \korean{있어} & \korean{있어요} & \korean{있습니다} \\
         &  &  \korean{있+다} & \korean{있+어} & \korean{있+어+요} & \korean{있+습니다}\\
         &  &  -\textsc{da}$_7$            & -\textsc{eo}$_7$            & -\textsc{eo}$_7$-\textsc{yo}            & -\textsc{p}$_5$-\textsc{ni}$_6$-\textsc{da}$_7$ \\
         &  & +ef & +ef & +ef+jxf & +ef \\
         \hline
 & Past & \korean{있었다} & \korean{있었어} & \korean{있었어요} & \korean{있었습니다} \\
         &  & \korean{있+었+다} & \korean{있+었+어} & \korean{있+었+어+요} & \korean{있+었+습니다}\\
         && -\textsc{ess}$_4$-\textsc{da}$_7$            & \textsc{ess}$_4$-\textsc{eo}$_7$             & \textsc{ess}$_4$-\textsc{eo}$_7$-\textsc{yo} & -\textsc{ess}$_4$-\textsc{p}$_5$-\textsc{ni}$_6$-\textsc{da}$_7$ \\
         &  & +ep+ef & +ep+ef & +ep+ef+jxf & +ep+ef \\
         \hline
\multirow{6}{*}{Interrogative} & Non-past  & \korean{있느냐} & \korean{있어} & \korean{있어요} & \korean{있습니까} \\
&& -\textsc{nya}$_7$ & -\textsc{eo}$_7$ & -\textsc{eo}$_7$-\textsc{yo} & -\textsc{p}$_5$-\textsc{ni}$_6$-\textsc{kka}$_7$ \\
         &  & \korean{있+느냐} & \korean{있+어} & \korean{있+어+요} & \korean{있+습니까}\\
         &  & +ef              & +ef  & +ef+jxf & +ef\\
         \hline
         & Past & \korean{있었느냐} & \korean{있었어} & \korean{있었어요} & \korean{있었습니까} \\
         && -\textsc{ess}$_4$-\textsc{nya}$_7$ & -\textsc{ess}$_4$-\textsc{eo}$_7$ & -\textsc{ess}$_4$-\textsc{eo}$_7$-\textsc{yo} & -\textsc{ess}$_4$-\textsc{p}$_5$-\textsc{ni}$_6$-\textsc{kka}$_7$ \\
         &  & \korean{있+었+느냐} & \korean{있+었+어} & \korean{있+었+어+요}\\
         &  & +ep+ef & paa+ep+ef &  paa+ep+ef+jxf\\
         \hline
\multirow{3}{*}{Assertive} &  & \korean{있겠다} & \korean{있겠어} & \korean{있겠어요} & \korean{있겠습니다} \\
&& -\textsc{get}$_4$-\textsc{da}$_7$ & -\textsc{get}$_4$-\textsc{eo}$_7$ & -\textsc{get}$_4$-\textsc{eo}$_7$-\textsc{yo} & -\textsc{get}$_4$-\textsc{p}$_5$-\textsc{ni}$_6$-\textsc{da}$_7$ \\
         &  & \korean{있+겠+다} & \korean{} & \korean{있+겠+어+요} & \korean{있+겠+습니다}\\
         &  & +ep+f             &           & paa+ep+ef+jxf & paa+ep+ef\\
    \end{tabular}
    \caption{Conjugation of \textit{itda}. Refer to Table~\ref{tab:korean-hada-1} for details.}
    \label{tab:kroean-itda}
\end{table}



%\begin{longtable}{lllll}
%NOMINALIZER-future-determiner-\korean{을} & EUL &	 128 \\
%NOMINALIZER-topic?past-determiner-\korean{은} & EUN&	 129 \\
%NOMINALIZER-nominalizer-informal-nonpolite-gi & GI &	 151 \\
%NOMINALIZER-nominalizer-formal-nonpolite-m 	& M & 216 \\
%\end{longtable}



